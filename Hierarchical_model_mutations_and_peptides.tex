\documentclass[]{article}
\usepackage{lmodern}
\usepackage{amssymb,amsmath}
\usepackage{ifxetex,ifluatex}
\usepackage{fixltx2e} % provides \textsubscript
\ifnum 0\ifxetex 1\fi\ifluatex 1\fi=0 % if pdftex
  \usepackage[T1]{fontenc}
  \usepackage[utf8]{inputenc}
\else % if luatex or xelatex
  \ifxetex
    \usepackage{mathspec}
    \usepackage{xltxtra,xunicode}
  \else
    \usepackage{fontspec}
  \fi
  \defaultfontfeatures{Mapping=tex-text,Scale=MatchLowercase}
  \newcommand{\euro}{€}
\fi
% use upquote if available, for straight quotes in verbatim environments
\IfFileExists{upquote.sty}{\usepackage{upquote}}{}
% use microtype if available
\IfFileExists{microtype.sty}{%
\usepackage{microtype}
\UseMicrotypeSet[protrusion]{basicmath} % disable protrusion for tt fonts
}{}
\usepackage[margin=1in]{geometry}
\usepackage{color}
\usepackage{fancyvrb}
\newcommand{\VerbBar}{|}
\newcommand{\VERB}{\Verb[commandchars=\\\{\}]}
\DefineVerbatimEnvironment{Highlighting}{Verbatim}{commandchars=\\\{\}}
% Add ',fontsize=\small' for more characters per line
\usepackage{framed}
\definecolor{shadecolor}{RGB}{248,248,248}
\newenvironment{Shaded}{\begin{snugshade}}{\end{snugshade}}
\newcommand{\KeywordTok}[1]{\textcolor[rgb]{0.13,0.29,0.53}{\textbf{{#1}}}}
\newcommand{\DataTypeTok}[1]{\textcolor[rgb]{0.13,0.29,0.53}{{#1}}}
\newcommand{\DecValTok}[1]{\textcolor[rgb]{0.00,0.00,0.81}{{#1}}}
\newcommand{\BaseNTok}[1]{\textcolor[rgb]{0.00,0.00,0.81}{{#1}}}
\newcommand{\FloatTok}[1]{\textcolor[rgb]{0.00,0.00,0.81}{{#1}}}
\newcommand{\CharTok}[1]{\textcolor[rgb]{0.31,0.60,0.02}{{#1}}}
\newcommand{\StringTok}[1]{\textcolor[rgb]{0.31,0.60,0.02}{{#1}}}
\newcommand{\CommentTok}[1]{\textcolor[rgb]{0.56,0.35,0.01}{\textit{{#1}}}}
\newcommand{\OtherTok}[1]{\textcolor[rgb]{0.56,0.35,0.01}{{#1}}}
\newcommand{\AlertTok}[1]{\textcolor[rgb]{0.94,0.16,0.16}{{#1}}}
\newcommand{\FunctionTok}[1]{\textcolor[rgb]{0.00,0.00,0.00}{{#1}}}
\newcommand{\RegionMarkerTok}[1]{{#1}}
\newcommand{\ErrorTok}[1]{\textbf{{#1}}}
\newcommand{\NormalTok}[1]{{#1}}
\usepackage{graphicx}
\makeatletter
\def\maxwidth{\ifdim\Gin@nat@width>\linewidth\linewidth\else\Gin@nat@width\fi}
\def\maxheight{\ifdim\Gin@nat@height>\textheight\textheight\else\Gin@nat@height\fi}
\makeatother
% Scale images if necessary, so that they will not overflow the page
% margins by default, and it is still possible to overwrite the defaults
% using explicit options in \includegraphics[width, height, ...]{}
\setkeys{Gin}{width=\maxwidth,height=\maxheight,keepaspectratio}
\ifxetex
  \usepackage[setpagesize=false, % page size defined by xetex
              unicode=false, % unicode breaks when used with xetex
              xetex]{hyperref}
\else
  \usepackage[unicode=true]{hyperref}
\fi
\hypersetup{breaklinks=true,
            bookmarks=true,
            pdfauthor={Jacqueline Buros Novik},
            pdftitle={Hierarchical Analysis of Mutation Burden},
            colorlinks=true,
            citecolor=blue,
            urlcolor=blue,
            linkcolor=magenta,
            pdfborder={0 0 0}}
\urlstyle{same}  % don't use monospace font for urls
\setlength{\parindent}{0pt}
\setlength{\parskip}{6pt plus 2pt minus 1pt}
\setlength{\emergencystretch}{3em}  % prevent overfull lines
\setcounter{secnumdepth}{5}

%%% Use protect on footnotes to avoid problems with footnotes in titles
\let\rmarkdownfootnote\footnote%
\def\footnote{\protect\rmarkdownfootnote}

%%% Change title format to be more compact
\usepackage{titling}

% Create subtitle command for use in maketitle
\newcommand{\subtitle}[1]{
  \posttitle{
    \begin{center}\large#1\end{center}
    }
}

\setlength{\droptitle}{-2em}
  \title{Hierarchical Analysis of Mutation Burden}
  \pretitle{\vspace{\droptitle}\centering\huge}
  \posttitle{\par}
  \author{Jacqueline Buros Novik}
  \preauthor{\centering\large\emph}
  \postauthor{\par}
  \predate{\centering\large\emph}
  \postdate{\par}
  \date{7/5/2017}


\begin{document}

\maketitle


{
\hypersetup{linkcolor=black}
\setcounter{tocdepth}{2}
\tableofcontents
}
\section{Supplemental Notebook 1}\label{supplemental-notebook-1}

This is a re-analysis of mutation count by sample type (primary /
relapse), tissue type (solid vs ascites) and treatment exposure
(treatment naive vs treated).

This section uses a Bayesian analysis in order to better adjust for the
dramatic imbalance in groups.

\subsection{Summarize data}\label{summarize-data}

In this analysis we will be looking at how the number of
\texttt{mutations} \& \texttt{peptides} (predicted neoantigens) varies
with the sample type (solid / ascites) and timing of acquisition
(relapse / primary and treated / untreated).

\begin{Shaded}
\begin{Highlighting}[]
\KeywordTok{ggplot}\NormalTok{(md, }\KeywordTok{aes}\NormalTok{(}\DataTypeTok{x =} \NormalTok{mutations_per_mb, }\DataTypeTok{fill =} \NormalTok{specific_treatment)) +}\StringTok{ }
\StringTok{  }\KeywordTok{facet_wrap}\NormalTok{(~tissue_type) +}
\StringTok{  }\KeywordTok{geom_histogram}\NormalTok{(}\DataTypeTok{position =} \StringTok{'dodge'}\NormalTok{) +}
\StringTok{  }\KeywordTok{theme_minimal}\NormalTok{()}
\end{Highlighting}
\end{Shaded}

\begin{verbatim}
## `stat_bin()` using `bins = 30`. Pick better value with `binwidth`.
\end{verbatim}

\includegraphics{Hierarchical_model_mutations_and_peptides_files/figure-latex/plot-data-1.pdf}

How many observations do we have for each of these categories?

\begin{Shaded}
\begin{Highlighting}[]
\NormalTok{md %>%}
\StringTok{  }\KeywordTok{group_by}\NormalTok{(tissue_type, treatment, timepoint) %>%}
\StringTok{  }\KeywordTok{tally}\NormalTok{() %>%}
\StringTok{  }\NormalTok{tidyr::}\KeywordTok{spread}\NormalTok{(}\DataTypeTok{key =} \NormalTok{tissue_type, }\DataTypeTok{value =} \NormalTok{n, }\DataTypeTok{fill =} \DecValTok{0}\NormalTok{)}
\end{Highlighting}
\end{Shaded}

\begin{verbatim}
## Source: local data frame [3 x 4]
## Groups: treatment [2]
## 
##         treatment  timepoint ascites solid
## *           <chr>      <chr>   <dbl> <dbl>
## 1   chemo treated    primary       0     5
## 2   chemo treated recurrence      24     6
## 3 treatment naive    primary       4    75
\end{verbatim}

Strikes me that the ``recurrent'' timepoint is problematic in this
analysis, since we don't have any untreated/recurent samples \& so
cannot separate effect of recurrence from that of treatment.

Instead, we can look at \texttt{solid} samples only, among those
collected at the primary timepoint comparing treated to untreated
samples. My guess is, many of these might be the paired samples.

We can then include the treated / recurrence using only solid samples,
although my guess is in this case we will see a higher rate of mutations
among recurrence samples than among primary/treated.

Only after these effects are well established should we turn to the
ascites samples, to see if the difference between untreated/primary \&
treated/relapse is consisent with that seen in solid samples.

\subsection{Restricting to primary, solid
samples}\label{restricting-to-primary-solid-samples}

We now have samples, are treatment-naive.

Let's review how these metrics are distributed in this subset of our
samples. First we note that the maximum number of samples per donor in
this subset of our data is , meaning we have no duplicate samples.

Here, looking at metrics among all primary, solid samples irrespective
of treatment:

\begin{Shaded}
\begin{Highlighting}[]
\KeywordTok{ggplot}\NormalTok{(md_primary_solid %>%}
\StringTok{         }\NormalTok{tidyr::}\KeywordTok{gather}\NormalTok{(}\DataTypeTok{value =} \StringTok{'value'}\NormalTok{, }\DataTypeTok{key =} \StringTok{'variable'}\NormalTok{, mutations, mutations_per_mb, peptides), }\KeywordTok{aes}\NormalTok{(}\DataTypeTok{x =} \NormalTok{value)) +}\StringTok{ }
\StringTok{  }\KeywordTok{geom_density}\NormalTok{() +}\StringTok{ }
\StringTok{  }\KeywordTok{theme_minimal}\NormalTok{() +}\StringTok{ }
\StringTok{  }\KeywordTok{facet_wrap}\NormalTok{(~variable, }\DataTypeTok{scale =} \StringTok{'free'}\NormalTok{)}
\end{Highlighting}
\end{Shaded}

\includegraphics{Hierarchical_model_mutations_and_peptides_files/figure-latex/psolid-review-dist-1.pdf}

These numbers are not exactly normally-distributed.

Perhaps using a log-transformed value?

\begin{Shaded}
\begin{Highlighting}[]
\KeywordTok{ggplot}\NormalTok{(md_primary_solid %>%}
\StringTok{         }\NormalTok{tidyr::}\KeywordTok{gather}\NormalTok{(}\DataTypeTok{value =} \StringTok{'value'}\NormalTok{, }\DataTypeTok{key =} \StringTok{'variable'}\NormalTok{, mutations, mutations_per_mb, peptides),}
       \KeywordTok{aes}\NormalTok{(}\DataTypeTok{x =} \KeywordTok{log1p}\NormalTok{(value))) +}\StringTok{ }
\StringTok{  }\KeywordTok{geom_density}\NormalTok{() +}\StringTok{ }
\StringTok{  }\KeywordTok{theme_minimal}\NormalTok{() +}\StringTok{ }
\StringTok{  }\KeywordTok{facet_wrap}\NormalTok{(~variable, }\DataTypeTok{scale =} \StringTok{'free'}\NormalTok{)}
\end{Highlighting}
\end{Shaded}

\includegraphics{Hierarchical_model_mutations_and_peptides_files/figure-latex/psolid-review-dist-log-1.pdf}

\begin{Shaded}
\begin{Highlighting}[]
\KeywordTok{ggplot}\NormalTok{(md_primary_solid %>%}
\StringTok{         }\NormalTok{tidyr::}\KeywordTok{gather}\NormalTok{(}\DataTypeTok{value =} \StringTok{'value'}\NormalTok{, }\DataTypeTok{key =} \StringTok{'variable'}\NormalTok{, mutations, mutations_per_mb, peptides),}
       \KeywordTok{aes}\NormalTok{(}\DataTypeTok{x =} \KeywordTok{log1p}\NormalTok{(value), }\DataTypeTok{fill =} \NormalTok{treatment)) +}\StringTok{ }
\StringTok{  }\KeywordTok{geom_density}\NormalTok{(}\DataTypeTok{alpha =} \FloatTok{0.4}\NormalTok{) +}\StringTok{ }
\StringTok{  }\KeywordTok{theme_minimal}\NormalTok{() +}\StringTok{ }
\StringTok{  }\KeywordTok{facet_wrap}\NormalTok{(~variable, }\DataTypeTok{scale =} \StringTok{'free'}\NormalTok{)}
\end{Highlighting}
\end{Shaded}

\includegraphics{Hierarchical_model_mutations_and_peptides_files/figure-latex/psolid-review-dist-log-by-trt-1.pdf}

What is noticeable here is that, given the small number of treated
samples, it is very hard to tell graphically whether there is any
difference in the two distributions.

Let's try fitting a model to these data.

\begin{Shaded}
\begin{Highlighting}[]
\NormalTok{trt1 <-}\StringTok{ }\NormalTok{rstanarm::}\KeywordTok{stan_glm}\NormalTok{(}\KeywordTok{log1p}\NormalTok{(mutations) ~}\StringTok{ }\NormalTok{treatment,}
                           \DataTypeTok{data =} \NormalTok{md_primary_solid,}
                           \DataTypeTok{seed =} \NormalTok{stan_seed}
                           \NormalTok{)}
\NormalTok{trt1}
\end{Highlighting}
\end{Shaded}

\begin{verbatim}
## stan_glm
##  family:  gaussian [identity]
##  formula: log1p(mutations) ~ treatment
## ------
## 
## Estimates:
##                          Median MAD_SD
## (Intercept)              8.7    0.2   
## treatmenttreatment naive 0.2    0.2   
## sigma                    0.5    0.0   
## 
## Sample avg. posterior predictive 
## distribution of y (X = xbar):
##          Median MAD_SD
## mean_PPD 8.8    0.1   
## 
## ------
## For info on the priors used see help('prior_summary.stanreg').
\end{verbatim}

This suggests the treatment effect on number of mutations may be
relatively modest, with a median effect indicating that the average
mutation count among treatment naive samples would be 20\% higher than
that among chemo-treated samples (with a relatively wide posterior
interval).

\begin{Shaded}
\begin{Highlighting}[]
\NormalTok{bayesplot::}\KeywordTok{mcmc_areas}\NormalTok{(}\KeywordTok{as.array}\NormalTok{(trt1), }\DataTypeTok{pars =} \StringTok{'treatmenttreatment naive'}\NormalTok{)}
\end{Highlighting}
\end{Shaded}

\includegraphics{Hierarchical_model_mutations_and_peptides_files/figure-latex/psolid-trt1-coef-plot-1.pdf}

How well do this model's predictions match our data?

\begin{Shaded}
\begin{Highlighting}[]
\NormalTok{trt1.ppred <-}\StringTok{ }\NormalTok{rstanarm::}\KeywordTok{predictive_interval}\NormalTok{(trt1) %>%}
\StringTok{  }\KeywordTok{tbl_df}\NormalTok{(.)}
\NormalTok{trt1.median <-}\StringTok{ }\NormalTok{rstanarm::}\KeywordTok{predictive_interval}\NormalTok{(trt1, }\FloatTok{0.01}\NormalTok{) %>%}
\StringTok{  }\KeywordTok{tbl_df}\NormalTok{(.) %>%}
\StringTok{  }\NormalTok{dplyr::}\KeywordTok{mutate}\NormalTok{(}\DataTypeTok{median =} \NormalTok{(}\StringTok{`}\DataTypeTok{49.5%}\StringTok{`} \NormalTok{+}\StringTok{ `}\DataTypeTok{50.5%}\StringTok{`}\NormalTok{)/}\DecValTok{2}\NormalTok{) %>%}
\StringTok{  }\NormalTok{dplyr::}\KeywordTok{select}\NormalTok{(median)}

\NormalTok{md_primary_solid2 <-}\StringTok{ }
\StringTok{  }\NormalTok{md_primary_solid %>%}\StringTok{ }
\StringTok{  }\NormalTok{dplyr::}\KeywordTok{bind_cols}\NormalTok{(trt1.ppred) %>%}
\StringTok{  }\NormalTok{dplyr::}\KeywordTok{bind_cols}\NormalTok{(trt1.median)}
\end{Highlighting}
\end{Shaded}

\begin{Shaded}
\begin{Highlighting}[]
\KeywordTok{ggplot}\NormalTok{(md_primary_solid2, }\KeywordTok{aes}\NormalTok{(}\DataTypeTok{x =} \NormalTok{treatment, }\DataTypeTok{y =} \KeywordTok{log1p}\NormalTok{(mutations))) +}\StringTok{ }
\StringTok{  }\KeywordTok{geom_jitter}\NormalTok{() +}
\StringTok{  }\KeywordTok{geom_errorbar}\NormalTok{(}\KeywordTok{aes}\NormalTok{(}\DataTypeTok{x =} \NormalTok{treatment, }\DataTypeTok{ymin =} \StringTok{`}\DataTypeTok{5%}\StringTok{`}\NormalTok{, }\DataTypeTok{ymax =} \StringTok{`}\DataTypeTok{95%}\StringTok{`}\NormalTok{),}
                \DataTypeTok{data =} \NormalTok{md_primary_solid2 %>%}\StringTok{ }\NormalTok{dplyr::}\KeywordTok{distinct}\NormalTok{(treatment, }\DataTypeTok{.keep_all=}\NormalTok{T),}
                \DataTypeTok{colour =} \StringTok{'red'}\NormalTok{, }\DataTypeTok{alpha =} \FloatTok{0.5}\NormalTok{)}
\end{Highlighting}
\end{Shaded}

\includegraphics{Hierarchical_model_mutations_and_peptides_files/figure-latex/psolid-trt1-ppred-1.pdf}

How well does our model recover the observed distributions of variables?

\begin{Shaded}
\begin{Highlighting}[]
\NormalTok{bayesplot::}\KeywordTok{pp_check}\NormalTok{(trt1)}
\end{Highlighting}
\end{Shaded}

\includegraphics{Hierarchical_model_mutations_and_peptides_files/figure-latex/psolid-trt1-ppcheck-1.pdf}

Not bad ..

\subsubsection{Try a negative-binomial
model?}\label{try-a-negative-binomial-model}

What if we tried a negative-binomial model instead?

\begin{Shaded}
\begin{Highlighting}[]
\NormalTok{trt1nb <-}\StringTok{ }\NormalTok{rstanarm::}\KeywordTok{stan_glm}\NormalTok{(mutations ~}\StringTok{ }\NormalTok{treatment,}
                             \DataTypeTok{data =} \NormalTok{md_primary_solid,}
                             \DataTypeTok{family =} \KeywordTok{neg_binomial_2}\NormalTok{(),}
                             \DataTypeTok{seed =} \NormalTok{stan_seed}
\NormalTok{)}
\NormalTok{trt1nb}
\end{Highlighting}
\end{Shaded}

\begin{verbatim}
## stan_glm
##  family:  neg_binomial_2 [log]
##  formula: mutations ~ treatment
## ------
## 
## Estimates:
##                          Median MAD_SD
## (Intercept)              8.8    0.2   
## treatmenttreatment naive 0.2    0.2   
## reciprocal_dispersion    4.3    0.7   
## 
## Sample avg. posterior predictive 
## distribution of y (X = xbar):
##          Median MAD_SD
## mean_PPD 7777.7  612.2
## 
## ------
## For info on the priors used see help('prior_summary.stanreg').
\end{verbatim}

(notice that here we have almost identical parameter estimates)

\begin{Shaded}
\begin{Highlighting}[]
\NormalTok{bayesplot::}\KeywordTok{pp_check}\NormalTok{(trt1nb)}
\end{Highlighting}
\end{Shaded}

\includegraphics{Hierarchical_model_mutations_and_peptides_files/figure-latex/psolid-trt1nb-ppcheck-1.pdf}

Here we have a slightly better fit, but not by much. Consistent with
theory, the log-transform works well as an approximation to the
`counting process' at high levels of the counts.

\subsubsection{Adjust for number of
cycles?}\label{adjust-for-number-of-cycles}

Next we look at estimating the effects of number of cycles on mutation
count. Sometimes adding more information can address noise in the model,
and sometimes it just .. adds noise.

\begin{Shaded}
\begin{Highlighting}[]
\NormalTok{trt2 <-}\StringTok{ }\NormalTok{rstanarm::}\KeywordTok{stan_glm}\NormalTok{(}\KeywordTok{log1p}\NormalTok{(mutations) ~}\StringTok{ }\NormalTok{treatment +}\StringTok{ `}\DataTypeTok{total cycles}\StringTok{`}\NormalTok{,}
                           \DataTypeTok{data =} \NormalTok{md_primary_solid %>%}
\StringTok{                             }\NormalTok{dplyr::}\KeywordTok{mutate}\NormalTok{(}\DataTypeTok{no_treatment =} \KeywordTok{ifelse}\NormalTok{(treatment ==}\StringTok{ 'treatment naive'}\NormalTok{, }\DecValTok{1}\NormalTok{, }\DecValTok{0}\NormalTok{),}
                                           \DataTypeTok{treatment =} \KeywordTok{ifelse}\NormalTok{(treatment !=}\StringTok{ 'treatment naive'}\NormalTok{, }\DecValTok{1}\NormalTok{, }\DecValTok{0}\NormalTok{))}
                           \NormalTok{)}
\NormalTok{trt2}
\end{Highlighting}
\end{Shaded}

\begin{verbatim}
## stan_glm
##  family:  gaussian [identity]
##  formula: log1p(mutations) ~ treatment + `total cycles`
## ------
## 
## Estimates:
##                Median MAD_SD
## (Intercept)     8.9    0.1  
## treatment       1.0    0.8  
## `total cycles` -0.2    0.1  
## sigma           0.5    0.0  
## 
## Sample avg. posterior predictive 
## distribution of y (X = xbar):
##          Median MAD_SD
## mean_PPD 8.8    0.1   
## 
## ------
## For info on the priors used see help('prior_summary.stanreg').
\end{verbatim}

Let's plot the distributions around these effects

\begin{Shaded}
\begin{Highlighting}[]
\NormalTok{bayesplot::}\KeywordTok{mcmc_areas}\NormalTok{(}\KeywordTok{as.array}\NormalTok{(trt2), }\DataTypeTok{pars =} \KeywordTok{c}\NormalTok{(}\StringTok{'treatment'}\NormalTok{, }\StringTok{'`total cycles`'}\NormalTok{))}
\end{Highlighting}
\end{Shaded}

\includegraphics{Hierarchical_model_mutations_and_peptides_files/figure-latex/psolid-trt2-coef-plot-1.pdf}

The interpretation of these results would be that :

\begin{enumerate}
\def\labelenumi{\arabic{enumi}.}
\itemsep1pt\parskip0pt\parsep0pt
\item
  Samples that received treatment have higher average mutation count
  than samples that are treatment-naive
\item
  Among those receiving treatment, those with more cycles tended to have
  lower mutation count
\end{enumerate}

This may or may not make biological sense (to me it feels like a
stretch), and the posterior distributions of effects are all pretty
broad. So my inclination would be to judge these effects as being
``within the noise''.

\subsubsection{Use a varying-intercept
effect?}\label{use-a-varying-intercept-effect}

Before moving on to include solid/relapse/treated \& ascites samples, we
fit the same model using a varying-coefficient structure. We do this
because this is the type of model we will fit in later iterations and so
it will be helpful to have a baseline here to compare against.

This model is structurally very similar to fitting a no-intercept model,
with formula \texttt{mutations \textasciitilde{} 0 + treatment}, which
would estimate separate mean numbers of mutations among treated \& naive
solid primary samples. The difference is that, in this formulation, the
treatment-specific means are drawn from a higher-level distribution of
means which acts like a prior for the group-specific means, and which in
effect regularizes the treatment \& non-treatment means -- the
group-specific means end up shrinking towards an overall inter-group
mean.

\begin{Shaded}
\begin{Highlighting}[]
\NormalTok{trt3 <-}\StringTok{ }\NormalTok{rstanarm::}\KeywordTok{stan_glmer}\NormalTok{(}\KeywordTok{log1p}\NormalTok{(mutations) ~}\StringTok{ }\NormalTok{(}\DecValTok{1} \NormalTok{|}\StringTok{ }\NormalTok{treatment),}
                           \DataTypeTok{data =} \NormalTok{md_primary_solid,}
                           \DataTypeTok{adapt_delta =} \FloatTok{0.999}\NormalTok{,}
                           \DataTypeTok{seed =} \NormalTok{stan_seed}
                           \NormalTok{)}
\NormalTok{trt3}
\end{Highlighting}
\end{Shaded}

\begin{verbatim}
## stan_glmer
##  family:  gaussian [identity]
##  formula: log1p(mutations) ~ (1 | treatment)
## ------
## 
## Estimates:
##             Median MAD_SD
## (Intercept) 8.8    0.2   
## sigma       0.5    0.0   
## 
## Error terms:
##  Groups    Name        Std.Dev.
##  treatment (Intercept) 0.42    
##  Residual              0.48    
## Num. levels: treatment 2 
## 
## Sample avg. posterior predictive 
## distribution of y (X = xbar):
##          Median MAD_SD
## mean_PPD 8.8    0.1   
## 
## ------
## For info on the priors used see help('prior_summary.stanreg').
\end{verbatim}

The summary for this model does not include the group-specific means by
default. We can, however, recover them quite easily.

\begin{Shaded}
\begin{Highlighting}[]
\NormalTok{bayesplot::}\KeywordTok{mcmc_areas}\NormalTok{(}\KeywordTok{as.array}\NormalTok{(trt3), }\DataTypeTok{regex_pars =} \StringTok{'}\CharTok{\textbackslash{}\textbackslash{}}\StringTok{(Intercept}\CharTok{\textbackslash{}\textbackslash{}}\StringTok{) treatment}\CharTok{\textbackslash{}\textbackslash{}}\StringTok{:'}\NormalTok{)}
\end{Highlighting}
\end{Shaded}

\includegraphics{Hierarchical_model_mutations_and_peptides_files/figure-latex/psolid-trt3-coef-plot-1.pdf}

A textual summary can be accessed via \texttt{summary}:

\begin{Shaded}
\begin{Highlighting}[]
\KeywordTok{summary}\NormalTok{(trt3, }\DataTypeTok{regex_pars =} \KeywordTok{c}\NormalTok{(}\StringTok{'^b'}\NormalTok{))}
\end{Highlighting}
\end{Shaded}

\begin{verbatim}
## 
## Model Info:
## 
##  function:  stan_glmer
##  family:    gaussian [identity]
##  formula:   log1p(mutations) ~ (1 | treatment)
##  algorithm: sampling
##  priors:    see help('prior_summary')
##  sample:    4000 (posterior sample size)
##  num obs:   80
##  groups:    treatment (2)
## 
## Estimates:
##                                            mean   sd   2.5%   25%   50%
## b[(Intercept) treatment:chemo_treated]    0.0    0.3 -0.6   -0.1   0.0 
## b[(Intercept) treatment:treatment_naive]  0.1    0.3 -0.5    0.0   0.0 
##                                            75%   97.5%
## b[(Intercept) treatment:chemo_treated]    0.0   0.7   
## b[(Intercept) treatment:treatment_naive]  0.2   0.9   
## 
## Diagnostics:
##                                          mcse Rhat n_eff
## b[(Intercept) treatment:chemo_treated]   0.0  1.0  730  
## b[(Intercept) treatment:treatment_naive] 0.0  1.0  594  
## 
## For each parameter, mcse is Monte Carlo standard error, n_eff is a crude measure of effective sample size, and Rhat is the potential scale reduction factor on split chains (at convergence Rhat=1).
\end{verbatim}

However, in order to estimate the average difference between treated and
naive samples, it is easiest to use posterior-predicted values.

For this, we construct a hypothetical dataset containing one treated \&
one treatment-naive sample.

\begin{Shaded}
\begin{Highlighting}[]
\NormalTok{newdata <-}\StringTok{ }\KeywordTok{data.frame}\NormalTok{(}\DataTypeTok{treatment =} \KeywordTok{c}\NormalTok{(}\StringTok{'chemo_treated'}\NormalTok{, }\StringTok{'treatment_naive'}\NormalTok{),}
                      \DataTypeTok{timepoint =} \KeywordTok{c}\NormalTok{(}\StringTok{'primary'}\NormalTok{, }\StringTok{'primary'}\NormalTok{))}
\NormalTok{ppred_newdata <-}\StringTok{ }\NormalTok{rstanarm::}\KeywordTok{posterior_predict}\NormalTok{(trt3, }\DataTypeTok{newdata =} \NormalTok{newdata)}
\KeywordTok{summary}\NormalTok{(}\KeywordTok{apply}\NormalTok{(ppred_newdata, }\DecValTok{1}\NormalTok{, diff))}
\end{Highlighting}
\end{Shaded}

\begin{verbatim}
##     Min.  1st Qu.   Median     Mean  3rd Qu.     Max. 
## -2.93000 -0.36350  0.11370  0.09895  0.55950  2.82300
\end{verbatim}

This yields the posterior-predicted distribution of the difference (on
scale of \texttt{log1(mutations)}) between chemo-treated \&
treatment-naive samples.

It's not surprising, here, that our effect has shrunken -- the mean
effect of having a treatment\_naive sample is 0.12 (or 12\% higher
mutation count than chemo-treated samples) instead of 0.2. Our
confidence intervals around this difference is also much greater than in
the previous model. However the direction of effect is similar. Folks
may reasonably disagree about which of these two results is more
``correct'', but the findings aren't inconsistent.

At this point, we are ready to move on to a variation of this model that
includes all solid samples, estimating effects of relapse \& treatment
separately.

\subsection{All solid tumor samples}\label{all-solid-tumor-samples}

Now we start to analyze a dataset including primary/untreated,
primary/treated \& relapse/treated samples.

First we note that the maximum number of samples per donor in this
subset of our data is , meaning we have a handful of duplicate samples
per donor. We will adjust for this in our analysis.

\subsubsection{Using a standard Bayesian
glm}\label{using-a-standard-bayesian-glm}

First we fit a standard \texttt{glm} without any donor-specific
adjustments.

\begin{Shaded}
\begin{Highlighting}[]
\NormalTok{strt1 <-}\StringTok{ }\NormalTok{rstanarm::}\KeywordTok{stan_glm}\NormalTok{(}\KeywordTok{log1p}\NormalTok{(mutations) ~}\StringTok{ }\NormalTok{treatment +}\StringTok{ }\NormalTok{timepoint,}
                           \DataTypeTok{data =} \NormalTok{md_solid, }
                           \DataTypeTok{adapt_delta =} \FloatTok{0.999}\NormalTok{,}
                           \DataTypeTok{seed =} \NormalTok{stan_seed}
                           \NormalTok{)}
\NormalTok{strt1}
\end{Highlighting}
\end{Shaded}

\begin{verbatim}
## stan_glm
##  family:  gaussian [identity]
##  formula: log1p(mutations) ~ treatment + timepoint
## ------
## 
## Estimates:
##                          Median MAD_SD
## (Intercept)              8.7    0.2   
## treatmenttreatment naive 0.2    0.2   
## timepointrecurrence      0.8    0.3   
## sigma                    0.5    0.0   
## 
## Sample avg. posterior predictive 
## distribution of y (X = xbar):
##          Median MAD_SD
## mean_PPD 8.9    0.1   
## 
## ------
## For info on the priors used see help('prior_summary.stanreg').
\end{verbatim}

Here, we see a similar treatment effect as in our earlier analysis
(which is encouraging), but the estimated ``recurrence'' effect is
somewhat higher than the treatment effect.

How well are we recovering the distribution of our
\texttt{log1p(mutation)} count in this sample?

\begin{Shaded}
\begin{Highlighting}[]
\NormalTok{bayesplot::}\KeywordTok{pp_check}\NormalTok{(strt1)}
\end{Highlighting}
\end{Shaded}

\includegraphics{Hierarchical_model_mutations_and_peptides_files/figure-latex/allsolid-strt1-ppcheck-1.pdf}

Pretty well.

\subsubsection{Using a varying-intercept
model}\label{using-a-varying-intercept-model}

Let's fit this model with an adjustment for within-id similarity.

\begin{Shaded}
\begin{Highlighting}[]
\NormalTok{strt2 <-}\StringTok{ }\NormalTok{rstanarm::}\KeywordTok{stan_glmer}\NormalTok{(}\KeywordTok{log1p}\NormalTok{(mutations) ~}\StringTok{ }\NormalTok{treatment +}\StringTok{ }\NormalTok{timepoint +}\StringTok{ }\NormalTok{(}\DecValTok{1} \NormalTok{|}\StringTok{ }\NormalTok{donor),}
                           \DataTypeTok{data =} \NormalTok{md_solid, }
                           \DataTypeTok{adapt_delta =} \FloatTok{0.999}\NormalTok{,}
                           \DataTypeTok{iter =} \DecValTok{5000}\NormalTok{,}
                           \DataTypeTok{seed =} \NormalTok{stan_seed}
                           \NormalTok{)}
\end{Highlighting}
\end{Shaded}

\begin{verbatim}
## Warning: There were 4 chains where the estimated Bayesian Fraction of Missing Information was low. See
## http://mc-stan.org/misc/warnings.html#bfmi-low
\end{verbatim}

\begin{verbatim}
## Warning: Examine the pairs() plot to diagnose sampling problems
\end{verbatim}

\begin{Shaded}
\begin{Highlighting}[]
\NormalTok{strt2}
\end{Highlighting}
\end{Shaded}

\begin{verbatim}
## stan_glmer
##  family:  gaussian [identity]
##  formula: log1p(mutations) ~ treatment + timepoint + (1 | donor)
## ------
## 
## Estimates:
##                          Median MAD_SD
## (Intercept)              8.7    0.2   
## treatmenttreatment naive 0.2    0.2   
## timepointrecurrence      0.5    0.3   
## sigma                    0.3    0.1   
## 
## Error terms:
##  Groups   Name        Std.Dev.
##  donor    (Intercept) 0.36    
##  Residual             0.29    
## Num. levels: donor 81 
## 
## Sample avg. posterior predictive 
## distribution of y (X = xbar):
##          Median MAD_SD
## mean_PPD 8.9    0.0   
## 
## ------
## For info on the priors used see help('prior_summary.stanreg').
\end{verbatim}

These findings are similar to those from the earlier model which did not
adjust for duplicate samples per donor, although the effect is somewhat
attenuated.

\begin{Shaded}
\begin{Highlighting}[]
\NormalTok{bayesplot::}\KeywordTok{mcmc_areas}\NormalTok{(}\KeywordTok{as.array}\NormalTok{(strt2), }\DataTypeTok{regex_pars =} \KeywordTok{c}\NormalTok{(}\StringTok{'^treatment'}\NormalTok{, }\StringTok{'^timepoint'}\NormalTok{))}
\end{Highlighting}
\end{Shaded}

\includegraphics{Hierarchical_model_mutations_and_peptides_files/figure-latex/allsolid-strt2-coef-plot-1.pdf}

According to this model, we have an estimated average 50\% increase in
mutations for relapse samples, vs primary.

The 95\% credible intervals for this increase are:

\begin{Shaded}
\begin{Highlighting}[]
\NormalTok{rstanarm::}\KeywordTok{posterior_interval}\NormalTok{(strt2, }\DataTypeTok{prob =} \FloatTok{0.95}\NormalTok{, }\DataTypeTok{pars =} \KeywordTok{c}\NormalTok{(}\StringTok{'timepointrecurrence'}\NormalTok{))}
\end{Highlighting}
\end{Shaded}

\begin{verbatim}
##                            2.5%    97.5%
## timepointrecurrence -0.05066051 1.125876
\end{verbatim}

How much of this probability mass is \textless{} 0?

\begin{Shaded}
\begin{Highlighting}[]
\KeywordTok{mean}\NormalTok{(}\KeywordTok{sapply}\NormalTok{(}\KeywordTok{as.array}\NormalTok{(strt2)[,,}\StringTok{'timepointrecurrence'}\NormalTok{], }\DataTypeTok{FUN =} \NormalTok{function(x) x <=}\StringTok{ }\DecValTok{0}\NormalTok{))}
\end{Highlighting}
\end{Shaded}

\begin{verbatim}
## [1] 0.0383
\end{verbatim}

This number reflects the so-called `bayesian p-value', IE the posterior
probability of a relapse effect being \textless{}= 0. Note that this
would correspond to a one-sided p-value in traditional frequentist NHT
framework.

\emph{Side note: this adjustment for \texttt{donor} only accounts for
the fact that we'd expect two samples from the same donor to be more
similar from one another than two samples from different donors. It does
not estimate ``varying-coefficients'', ie relapse or treatment effects
cannot vary by donor. All effects are estimated as population averages.}

\subsubsection{Using a varying-coefficient
model}\label{using-a-varying-coefficient-model}

Next we fit a varying-coefficient model, which allows the relapse effect
to vary by donor, but models those donor-specific variances as
deviations from an overall relapse effect on mutation count.

\begin{Shaded}
\begin{Highlighting}[]
\NormalTok{strt3 <-}\StringTok{ }\NormalTok{rstanarm::}\KeywordTok{stan_glmer}\NormalTok{(}\KeywordTok{log1p}\NormalTok{(mutations) ~}\StringTok{ }\NormalTok{treatment +}\StringTok{ }\NormalTok{timepoint +}\StringTok{ }\NormalTok{(}\DecValTok{1} \NormalTok{+}\StringTok{ }\NormalTok{timepoint |}\StringTok{ }\NormalTok{donor),}
                           \DataTypeTok{data =} \NormalTok{md_solid, }
                           \DataTypeTok{adapt_delta =} \FloatTok{0.999}\NormalTok{,}
                           \DataTypeTok{seed =} \NormalTok{stan_seed}
                           \NormalTok{)}
\end{Highlighting}
\end{Shaded}

\begin{verbatim}
## Warning: There were 4 chains where the estimated Bayesian Fraction of Missing Information was low. See
## http://mc-stan.org/misc/warnings.html#bfmi-low
\end{verbatim}

\begin{verbatim}
## Warning: Examine the pairs() plot to diagnose sampling problems
\end{verbatim}

\begin{Shaded}
\begin{Highlighting}[]
\NormalTok{strt3}
\end{Highlighting}
\end{Shaded}

\begin{verbatim}
## stan_glmer
##  family:  gaussian [identity]
##  formula: log1p(mutations) ~ treatment + timepoint + (1 + timepoint | donor)
## ------
## 
## Estimates:
##                          Median MAD_SD
## (Intercept)              8.7    0.2   
## treatmenttreatment naive 0.2    0.2   
## timepointrecurrence      0.6    0.4   
## sigma                    0.3    0.2   
## 
## Error terms:
##  Groups   Name                Std.Dev. Corr 
##  donor    (Intercept)         0.33          
##           timepointrecurrence 0.31     -0.09
##  Residual                     0.32          
## Num. levels: donor 81 
## 
## Sample avg. posterior predictive 
## distribution of y (X = xbar):
##          Median MAD_SD
## mean_PPD 8.9    0.0   
## 
## ------
## For info on the priors used see help('prior_summary.stanreg').
\end{verbatim}

As in our previous examples, we can plot the posterior density of the
estimated treatment \& relapse effects.

\begin{Shaded}
\begin{Highlighting}[]
\NormalTok{bayesplot::}\KeywordTok{mcmc_areas}\NormalTok{(}\KeywordTok{as.array}\NormalTok{(strt3), }\DataTypeTok{regex_pars =} \KeywordTok{c}\NormalTok{(}\StringTok{'^treatment'}\NormalTok{, }\StringTok{'^timepoint'}\NormalTok{))}
\end{Highlighting}
\end{Shaded}

\includegraphics{Hierarchical_model_mutations_and_peptides_files/figure-latex/allsolid-strt3-coef-plot-1.pdf}

It is not surprising that these effects are similar to those estimated
by the previous model, since most of our donors have only 1 sample. But
it is reassuring to know that this particular modeling choice has little
impact on our findings.

Here, giving a textual summary again

\begin{Shaded}
\begin{Highlighting}[]
\NormalTok{rstanarm::}\KeywordTok{posterior_interval}\NormalTok{(strt3, }\DataTypeTok{prob =} \FloatTok{0.95}\NormalTok{, }\DataTypeTok{pars =} \KeywordTok{c}\NormalTok{(}\StringTok{'timepointrecurrence'}\NormalTok{))}
\end{Highlighting}
\end{Shaded}

\begin{verbatim}
##                           2.5%    97.5%
## timepointrecurrence -0.2124145 1.256275
\end{verbatim}

And, calculating the percentage of posterior density that is
\textless{}= 0.

\begin{Shaded}
\begin{Highlighting}[]
\KeywordTok{mean}\NormalTok{(}\KeywordTok{sapply}\NormalTok{(}\KeywordTok{as.array}\NormalTok{(strt3)[,,}\StringTok{'timepointrecurrence'}\NormalTok{], }\DataTypeTok{FUN =} \NormalTok{function(x) x <=}\StringTok{ }\DecValTok{0}\NormalTok{))}
\end{Highlighting}
\end{Shaded}

\begin{verbatim}
## [1] 0.06625
\end{verbatim}

Finally, we can plot our posterior-predicted intervals for the three
models thus far, overlaying them with the observed data.

This is a somewhat useful way to gut-check the model parameters.

\begin{Shaded}
\begin{Highlighting}[]
\CommentTok{# calculating the posterior-predicted values}
\NormalTok{strt3.ppred <-}\StringTok{ }\NormalTok{rstanarm::}\KeywordTok{predictive_interval}\NormalTok{(strt3) %>%}
\StringTok{  }\KeywordTok{tbl_df}\NormalTok{(.)}
\NormalTok{strt3.median <-}\StringTok{ }\NormalTok{rstanarm::}\KeywordTok{predictive_interval}\NormalTok{(strt3, }\FloatTok{0.01}\NormalTok{) %>%}
\StringTok{  }\KeywordTok{tbl_df}\NormalTok{(.) %>%}
\StringTok{  }\NormalTok{dplyr::}\KeywordTok{mutate}\NormalTok{(}\DataTypeTok{median =} \NormalTok{(}\StringTok{`}\DataTypeTok{49.5%}\StringTok{`} \NormalTok{+}\StringTok{ `}\DataTypeTok{50.5%}\StringTok{`}\NormalTok{)/}\DecValTok{2}\NormalTok{) %>%}
\StringTok{  }\NormalTok{dplyr::}\KeywordTok{select}\NormalTok{(median)}

\NormalTok{md_solid3 <-}\StringTok{ }
\StringTok{  }\NormalTok{md_solid %>%}\StringTok{ }
\StringTok{  }\NormalTok{dplyr::}\KeywordTok{bind_cols}\NormalTok{(strt3.ppred) %>%}
\StringTok{  }\NormalTok{dplyr::}\KeywordTok{bind_cols}\NormalTok{(strt3.median)}

\NormalTok{strt2.ppred <-}\StringTok{ }\NormalTok{rstanarm::}\KeywordTok{predictive_interval}\NormalTok{(strt2) %>%}
\StringTok{  }\KeywordTok{tbl_df}\NormalTok{(.)}
\NormalTok{strt2.median <-}\StringTok{ }\NormalTok{rstanarm::}\KeywordTok{predictive_interval}\NormalTok{(strt2, }\FloatTok{0.01}\NormalTok{) %>%}
\StringTok{  }\KeywordTok{tbl_df}\NormalTok{(.) %>%}
\StringTok{  }\NormalTok{dplyr::}\KeywordTok{mutate}\NormalTok{(}\DataTypeTok{median =} \NormalTok{(}\StringTok{`}\DataTypeTok{49.5%}\StringTok{`} \NormalTok{+}\StringTok{ `}\DataTypeTok{50.5%}\StringTok{`}\NormalTok{)/}\DecValTok{2}\NormalTok{) %>%}
\StringTok{  }\NormalTok{dplyr::}\KeywordTok{select}\NormalTok{(median)}

\NormalTok{md_solid2 <-}\StringTok{ }
\StringTok{  }\NormalTok{md_solid %>%}\StringTok{ }
\StringTok{  }\NormalTok{dplyr::}\KeywordTok{bind_cols}\NormalTok{(strt2.ppred) %>%}
\StringTok{  }\NormalTok{dplyr::}\KeywordTok{bind_cols}\NormalTok{(strt2.median)}
\end{Highlighting}
\end{Shaded}

\begin{Shaded}
\begin{Highlighting}[]
\NormalTok{## plotting posterior-predicted values, with observed datapoints}
\KeywordTok{ggplot}\NormalTok{(md_solid3, }\KeywordTok{aes}\NormalTok{(}\DataTypeTok{x =} \NormalTok{specific_treatment, }\DataTypeTok{y =} \KeywordTok{log1p}\NormalTok{(mutations))) +}\StringTok{ }
\StringTok{  }\KeywordTok{geom_jitter}\NormalTok{() +}
\StringTok{  }\KeywordTok{geom_errorbar}\NormalTok{(}\KeywordTok{aes}\NormalTok{(}\DataTypeTok{x =} \NormalTok{specific_treatment, }\DataTypeTok{ymin =} \StringTok{`}\DataTypeTok{5%}\StringTok{`}\NormalTok{, }\DataTypeTok{ymax =} \StringTok{`}\DataTypeTok{95%}\StringTok{`}\NormalTok{, }\DataTypeTok{colour =} \StringTok{'model: primary only'}\NormalTok{),}
                \DataTypeTok{data =} \NormalTok{md_primary_solid2 %>%}\StringTok{ }\NormalTok{dplyr::}\KeywordTok{distinct}\NormalTok{(specific_treatment, }\DataTypeTok{.keep_all=}\NormalTok{T),}
                \DataTypeTok{alpha =} \FloatTok{0.5}\NormalTok{) +}
\StringTok{  }\KeywordTok{geom_errorbar}\NormalTok{(}\KeywordTok{aes}\NormalTok{(}\DataTypeTok{x =} \NormalTok{specific_treatment, }\DataTypeTok{ymin =} \StringTok{`}\DataTypeTok{5%}\StringTok{`}\NormalTok{, }\DataTypeTok{ymax =} \StringTok{`}\DataTypeTok{95%}\StringTok{`}\NormalTok{, }\DataTypeTok{colour =} \StringTok{'model: varying-slope'}\NormalTok{),}
                \DataTypeTok{data =} \NormalTok{md_solid3 %>%}\StringTok{ }\NormalTok{dplyr::}\KeywordTok{distinct}\NormalTok{(specific_treatment, }\DataTypeTok{.keep_all=}\NormalTok{T),}
                \DataTypeTok{alpha =} \FloatTok{0.5}\NormalTok{) +}
\StringTok{  }\KeywordTok{geom_errorbar}\NormalTok{(}\KeywordTok{aes}\NormalTok{(}\DataTypeTok{x =} \NormalTok{specific_treatment, }\DataTypeTok{ymin =} \StringTok{`}\DataTypeTok{5%}\StringTok{`}\NormalTok{, }\DataTypeTok{ymax =} \StringTok{`}\DataTypeTok{95%}\StringTok{`}\NormalTok{, }\DataTypeTok{colour =} \StringTok{'model: varying-int'}\NormalTok{),}
                \DataTypeTok{data =} \NormalTok{md_solid2 %>%}\StringTok{ }\NormalTok{dplyr::}\KeywordTok{distinct}\NormalTok{(specific_treatment, }\DataTypeTok{.keep_all=}\NormalTok{T),}
                \DataTypeTok{alpha =} \FloatTok{0.5}\NormalTok{) +}
\StringTok{  }\KeywordTok{theme_minimal}\NormalTok{()}
\end{Highlighting}
\end{Shaded}

\includegraphics{Hierarchical_model_mutations_and_peptides_files/figure-latex/allsolid-strt3-ppred-1.pdf}

\subsection{Including ascites samples}\label{including-ascites-samples}

Finally we look at a model including ascites samples

\begin{Shaded}
\begin{Highlighting}[]
\NormalTok{atrt1 <-}\StringTok{ }\NormalTok{rstanarm::}\KeywordTok{stan_glmer}\NormalTok{(}\KeywordTok{log1p}\NormalTok{(mutations) ~}\StringTok{ }
\StringTok{                                }\NormalTok{timepoint +}\StringTok{ }\NormalTok{treatment +}\StringTok{ }\NormalTok{(}\DecValTok{1} \NormalTok{|}\StringTok{ }\NormalTok{donor) +}
\StringTok{                                }\NormalTok{(}\DecValTok{1} \NormalTok{+}\StringTok{ }\NormalTok{timepoint +}\StringTok{ }\NormalTok{treatment |}\StringTok{ }\NormalTok{tissue_type),}
                              \DataTypeTok{data =} \NormalTok{md,}
                              \DataTypeTok{seed =} \NormalTok{stan_seed,}
                              \DataTypeTok{adapt_delta =} \FloatTok{0.999}\NormalTok{,}
                              \DataTypeTok{iter =} \DecValTok{4000}
                              \NormalTok{)}
\end{Highlighting}
\end{Shaded}

\begin{verbatim}
## Warning: There were 5 divergent transitions after warmup. Increasing adapt_delta above 0.999 may help. See
## http://mc-stan.org/misc/warnings.html#divergent-transitions-after-warmup
\end{verbatim}

\begin{verbatim}
## Warning: There were 4 chains where the estimated Bayesian Fraction of Missing Information was low. See
## http://mc-stan.org/misc/warnings.html#bfmi-low
\end{verbatim}

\begin{verbatim}
## Warning: Examine the pairs() plot to diagnose sampling problems
\end{verbatim}

\begin{Shaded}
\begin{Highlighting}[]
\NormalTok{atrt1}
\end{Highlighting}
\end{Shaded}

\begin{verbatim}
## stan_glmer
##  family:  gaussian [identity]
##  formula: log1p(mutations) ~ timepoint + treatment + (1 | donor) + (1 + 
##     timepoint + treatment | tissue_type)
## ------
## 
## Estimates:
##                          Median MAD_SD
## (Intercept)              8.7    0.2   
## timepointrecurrence      0.6    0.2   
## treatmenttreatment naive 0.2    0.2   
## sigma                    0.2    0.0   
## 
## Error terms:
##  Groups      Name                     Std.Dev. Corr       
##  donor       (Intercept)              0.40                
##  tissue_type (Intercept)              0.15                
##              timepointrecurrence      0.14     -0.03      
##              treatmenttreatment naive 0.14     -0.03  0.08
##  Residual                             0.20                
## Num. levels: donor 92, tissue_type 2 
## 
## Sample avg. posterior predictive 
## distribution of y (X = xbar):
##          Median MAD_SD
## mean_PPD 9.0    0.0   
## 
## ------
## For info on the priors used see help('prior_summary.stanreg').
\end{verbatim}

It's worth noting that this model recovers the distribution of our
original data pretty well.

\begin{Shaded}
\begin{Highlighting}[]
\NormalTok{bayesplot::}\KeywordTok{pp_check}\NormalTok{(atrt1)}
\end{Highlighting}
\end{Shaded}

\includegraphics{Hierarchical_model_mutations_and_peptides_files/figure-latex/allsamp-atrt1-ppcheck-1.pdf}

And, samples efficiently

\begin{Shaded}
\begin{Highlighting}[]
\NormalTok{bayesplot::}\KeywordTok{mcmc_trace}\NormalTok{(}\KeywordTok{as.array}\NormalTok{(atrt1), }\DataTypeTok{regex_pars =} \KeywordTok{c}\NormalTok{(}\StringTok{'^treatment'}\NormalTok{, }\StringTok{'^timepoint'}\NormalTok{), }\DataTypeTok{facet_args =} \KeywordTok{list}\NormalTok{(}\DataTypeTok{ncol =} \DecValTok{1}\NormalTok{))}
\end{Highlighting}
\end{Shaded}

\includegraphics{Hierarchical_model_mutations_and_peptides_files/figure-latex/allsamp-atrt1-traceplot-1.pdf}

This model includes an overall estimate for treatment \& timepoint
effects, from which tissue-type-specific coeffients are drawn.

Here are the overall estimated effects.

\begin{Shaded}
\begin{Highlighting}[]
\NormalTok{bayesplot::}\KeywordTok{mcmc_areas}\NormalTok{(}\KeywordTok{as.array}\NormalTok{(atrt1), }\DataTypeTok{regex_pars =} \KeywordTok{c}\NormalTok{(}\StringTok{'^treatment'}\NormalTok{, }\StringTok{'^timepoint'}\NormalTok{))}
\end{Highlighting}
\end{Shaded}

\includegraphics{Hierarchical_model_mutations_and_peptides_files/figure-latex/allsamp-atrt1-coef-plot-1.pdf}

Again, they are directionally similar to the effects estimated in our
subset analyses.

Let's compare the numerical summaries:

\begin{Shaded}
\begin{Highlighting}[]
\NormalTok{rstanarm::}\KeywordTok{posterior_interval}\NormalTok{(atrt1, }\DataTypeTok{prob =} \FloatTok{0.95}\NormalTok{, }\DataTypeTok{regex_pars =} \KeywordTok{c}\NormalTok{(}\StringTok{'^timepoint'}\NormalTok{, }\StringTok{'^treatment'}\NormalTok{))}
\end{Highlighting}
\end{Shaded}

\begin{verbatim}
##                                2.5%    97.5%
## timepointrecurrence       0.1147481 1.062608
## treatmenttreatment naive -0.2621648 0.631993
\end{verbatim}

The estimate for recurrent samples is higher in this model than was
estimated among solid samples only.

\begin{Shaded}
\begin{Highlighting}[]
\NormalTok{rstanarm::}\KeywordTok{posterior_interval}\NormalTok{(strt2, }\DataTypeTok{prob =} \FloatTok{0.95}\NormalTok{, }\DataTypeTok{regex_pars =} \KeywordTok{c}\NormalTok{(}\StringTok{'^timepoint'}\NormalTok{, }\StringTok{'^treatment'}\NormalTok{))}
\end{Highlighting}
\end{Shaded}

\begin{verbatim}
##                                 2.5%     97.5%
## timepointrecurrence      -0.05066051 1.1258756
## treatmenttreatment naive -0.25801920 0.5895171
\end{verbatim}

How much of this probability mass is \textless{} 0?

\begin{Shaded}
\begin{Highlighting}[]
\KeywordTok{mean}\NormalTok{(}\KeywordTok{sapply}\NormalTok{(}\KeywordTok{as.array}\NormalTok{(atrt1)[,,}\StringTok{'timepointrecurrence'}\NormalTok{], }\DataTypeTok{FUN =} \NormalTok{function(x) x <=}\StringTok{ }\DecValTok{0}\NormalTok{))}
\end{Highlighting}
\end{Shaded}

\begin{verbatim}
## [1] 0.010625
\end{verbatim}

Let's look at the estimates per tissue type.

\begin{Shaded}
\begin{Highlighting}[]
\NormalTok{bayesplot::}\KeywordTok{mcmc_areas}\NormalTok{(}\KeywordTok{as.array}\NormalTok{(atrt1), }\DataTypeTok{regex_pars =} \KeywordTok{c}\NormalTok{(}\StringTok{'^b}\CharTok{\textbackslash{}\textbackslash{}}\StringTok{[timepoint'}\NormalTok{))}
\end{Highlighting}
\end{Shaded}

\includegraphics{Hierarchical_model_mutations_and_peptides_files/figure-latex/allsamp-atrt1-coefplot-timepoint-by-tissue-type-1.pdf}

Interesting that, in this model, there is not a lot of variance of
effects by tissue type.

Same goes for the treatment effect.

\begin{Shaded}
\begin{Highlighting}[]
\NormalTok{bayesplot::}\KeywordTok{mcmc_areas}\NormalTok{(}\KeywordTok{as.array}\NormalTok{(atrt1), }\DataTypeTok{regex_pars =} \KeywordTok{c}\NormalTok{(}\StringTok{'^b}\CharTok{\textbackslash{}\textbackslash{}}\StringTok{[treatment'}\NormalTok{))}
\end{Highlighting}
\end{Shaded}

\includegraphics{Hierarchical_model_mutations_and_peptides_files/figure-latex/allsamp-atrt1-coefplot-treatment-by-tissue-type-1.pdf}

This could be due to a lack of sufficient data by which to estimate the
variance, thus causing the model to fall back on default weakly
informative priors which assume little variance by tissue type.

Let's now summarize our posterior-predicted values as we did in earlier
models.

\begin{Shaded}
\begin{Highlighting}[]
\NormalTok{atrt1.ppred <-}\StringTok{ }\NormalTok{rstanarm::}\KeywordTok{predictive_interval}\NormalTok{(atrt1) %>%}
\StringTok{  }\KeywordTok{tbl_df}\NormalTok{(.)}
\NormalTok{atrt1.median <-}\StringTok{ }\NormalTok{rstanarm::}\KeywordTok{predictive_interval}\NormalTok{(atrt1, }\FloatTok{0.01}\NormalTok{) %>%}
\StringTok{  }\KeywordTok{tbl_df}\NormalTok{(.) %>%}
\StringTok{  }\NormalTok{dplyr::}\KeywordTok{mutate}\NormalTok{(}\DataTypeTok{median =} \NormalTok{(}\StringTok{`}\DataTypeTok{49.5%}\StringTok{`} \NormalTok{+}\StringTok{ `}\DataTypeTok{50.5%}\StringTok{`}\NormalTok{)/}\DecValTok{2}\NormalTok{) %>%}
\StringTok{  }\NormalTok{dplyr::}\KeywordTok{select}\NormalTok{(median)}

\NormalTok{md1 <-}\StringTok{ }
\StringTok{  }\NormalTok{md %>%}\StringTok{ }
\StringTok{  }\NormalTok{dplyr::}\KeywordTok{bind_cols}\NormalTok{(atrt1.ppred) %>%}
\StringTok{  }\NormalTok{dplyr::}\KeywordTok{bind_cols}\NormalTok{(atrt1.median)}
\end{Highlighting}
\end{Shaded}

\begin{Shaded}
\begin{Highlighting}[]
\NormalTok{## plotting posterior-predicted values, with observed datapoints}
\KeywordTok{ggplot}\NormalTok{(md1, }\KeywordTok{aes}\NormalTok{(}\DataTypeTok{x =} \NormalTok{specific_treatment, }\DataTypeTok{y =} \KeywordTok{log1p}\NormalTok{(mutations))) +}\StringTok{ }
\StringTok{  }\KeywordTok{geom_jitter}\NormalTok{() +}
\StringTok{  }\KeywordTok{facet_wrap}\NormalTok{(~tissue_type) +}
\StringTok{  }\KeywordTok{geom_errorbar}\NormalTok{(}\KeywordTok{aes}\NormalTok{(}\DataTypeTok{x =} \NormalTok{specific_treatment, }\DataTypeTok{ymin =} \StringTok{`}\DataTypeTok{5%}\StringTok{`}\NormalTok{, }\DataTypeTok{ymax =} \StringTok{`}\DataTypeTok{95%}\StringTok{`}\NormalTok{, }\DataTypeTok{colour =} \StringTok{'model: primary only'}\NormalTok{),}
                \DataTypeTok{data =} \NormalTok{md_primary_solid2 %>%}\StringTok{ }\NormalTok{dplyr::}\KeywordTok{distinct}\NormalTok{(specific_treatment, }\DataTypeTok{.keep_all=}\NormalTok{T),}
                \DataTypeTok{alpha =} \FloatTok{0.5}\NormalTok{) +}
\StringTok{  }\KeywordTok{geom_errorbar}\NormalTok{(}\KeywordTok{aes}\NormalTok{(}\DataTypeTok{x =} \NormalTok{specific_treatment, }\DataTypeTok{ymin =} \StringTok{`}\DataTypeTok{5%}\StringTok{`}\NormalTok{, }\DataTypeTok{ymax =} \StringTok{`}\DataTypeTok{95%}\StringTok{`}\NormalTok{, }\DataTypeTok{colour =} \StringTok{'model: primary solid'}\NormalTok{),}
                \DataTypeTok{data =} \NormalTok{md_solid2 %>%}\StringTok{ }\NormalTok{dplyr::}\KeywordTok{distinct}\NormalTok{(specific_treatment, }\DataTypeTok{.keep_all=}\NormalTok{T),}
                \DataTypeTok{alpha =} \FloatTok{0.5}\NormalTok{) +}
\StringTok{  }\KeywordTok{geom_errorbar}\NormalTok{(}\KeywordTok{aes}\NormalTok{(}\DataTypeTok{x =} \NormalTok{specific_treatment, }\DataTypeTok{ymin =} \StringTok{`}\DataTypeTok{5%}\StringTok{`}\NormalTok{, }\DataTypeTok{ymax =} \StringTok{`}\DataTypeTok{95%}\StringTok{`}\NormalTok{, }\DataTypeTok{colour =} \StringTok{'model: all samples'}\NormalTok{),}
                \DataTypeTok{data =} \NormalTok{md1 %>%}\StringTok{ }\NormalTok{dplyr::}\KeywordTok{distinct}\NormalTok{(specific_treatment, tissue_type, }\DataTypeTok{.keep_all=}\NormalTok{T),}
                \DataTypeTok{alpha =} \FloatTok{0.5}\NormalTok{) +}
\StringTok{  }\KeywordTok{theme_minimal}\NormalTok{()}
\end{Highlighting}
\end{Shaded}

\includegraphics{Hierarchical_model_mutations_and_peptides_files/figure-latex/allsamp-atrt1-ppred-1.pdf}

Strange how, with each additional sample type added to the model, the
estimated mean primary/treated \texttt{log1p(mutation)} count gets
lower.

\subsubsection{Removing treatment from the
model}\label{removing-treatment-from-the-model}

What happens if we remove treatment from the model, and instead compare
all relapse vs all primary (treated + untreated) samples?

\begin{Shaded}
\begin{Highlighting}[]
\NormalTok{atrt2 <-}\StringTok{ }\NormalTok{rstanarm::}\KeywordTok{stan_glmer}\NormalTok{(}\KeywordTok{log1p}\NormalTok{(mutations) ~}\StringTok{ }
\StringTok{                                }\NormalTok{timepoint +}\StringTok{ }\NormalTok{(}\DecValTok{1} \NormalTok{|}\StringTok{ }\NormalTok{donor) +}
\StringTok{                                }\NormalTok{(}\DecValTok{1} \NormalTok{+}\StringTok{ }\NormalTok{timepoint |}\StringTok{ }\NormalTok{tissue_type),}
                              \DataTypeTok{data =} \NormalTok{md,}
                              \DataTypeTok{seed =} \NormalTok{stan_seed,}
                              \DataTypeTok{adapt_delta =} \FloatTok{0.999}\NormalTok{,}
                              \DataTypeTok{iter =} \DecValTok{4000}
                              \NormalTok{)}
\end{Highlighting}
\end{Shaded}

\begin{verbatim}
## Warning: There were 1 divergent transitions after warmup. Increasing adapt_delta above 0.999 may help. See
## http://mc-stan.org/misc/warnings.html#divergent-transitions-after-warmup
\end{verbatim}

\begin{verbatim}
## Warning: There were 4 chains where the estimated Bayesian Fraction of Missing Information was low. See
## http://mc-stan.org/misc/warnings.html#bfmi-low
\end{verbatim}

\begin{verbatim}
## Warning: Examine the pairs() plot to diagnose sampling problems
\end{verbatim}

\begin{Shaded}
\begin{Highlighting}[]
\NormalTok{atrt2}
\end{Highlighting}
\end{Shaded}

\begin{verbatim}
## stan_glmer
##  family:  gaussian [identity]
##  formula: log1p(mutations) ~ timepoint + (1 | donor) + (1 + timepoint | 
##     tissue_type)
## ------
## 
## Estimates:
##                     Median MAD_SD
## (Intercept)         8.9    0.1   
## timepointrecurrence 0.4    0.1   
## sigma               0.2    0.0   
## 
## Error terms:
##  Groups      Name                Std.Dev. Corr
##  donor       (Intercept)         0.40         
##  tissue_type (Intercept)         0.18         
##              timepointrecurrence 0.17     0.03
##  Residual                        0.20         
## Num. levels: donor 92, tissue_type 2 
## 
## Sample avg. posterior predictive 
## distribution of y (X = xbar):
##          Median MAD_SD
## mean_PPD 9.0    0.0   
## 
## ------
## For info on the priors used see help('prior_summary.stanreg').
\end{verbatim}

\begin{Shaded}
\begin{Highlighting}[]
\NormalTok{atrt2.ppred <-}\StringTok{ }\NormalTok{rstanarm::}\KeywordTok{predictive_interval}\NormalTok{(atrt2) %>%}
\StringTok{  }\KeywordTok{tbl_df}\NormalTok{(.)}
\NormalTok{atrt2.median <-}\StringTok{ }\NormalTok{rstanarm::}\KeywordTok{predictive_interval}\NormalTok{(atrt2, }\FloatTok{0.01}\NormalTok{) %>%}
\StringTok{  }\KeywordTok{tbl_df}\NormalTok{(.) %>%}
\StringTok{  }\NormalTok{dplyr::}\KeywordTok{mutate}\NormalTok{(}\DataTypeTok{median =} \NormalTok{(}\StringTok{`}\DataTypeTok{49.5%}\StringTok{`} \NormalTok{+}\StringTok{ `}\DataTypeTok{50.5%}\StringTok{`}\NormalTok{)/}\DecValTok{2}\NormalTok{) %>%}
\StringTok{  }\NormalTok{dplyr::}\KeywordTok{select}\NormalTok{(median)}

\NormalTok{md2 <-}\StringTok{ }
\StringTok{  }\NormalTok{md %>%}\StringTok{ }
\StringTok{  }\NormalTok{dplyr::}\KeywordTok{bind_cols}\NormalTok{(atrt2.ppred) %>%}
\StringTok{  }\NormalTok{dplyr::}\KeywordTok{bind_cols}\NormalTok{(atrt2.median)}
\end{Highlighting}
\end{Shaded}

\begin{Shaded}
\begin{Highlighting}[]
\NormalTok{## plotting posterior-predicted values, with observed datapoints}
\KeywordTok{ggplot}\NormalTok{(md2, }\KeywordTok{aes}\NormalTok{(}\DataTypeTok{x =} \NormalTok{specific_treatment, }\DataTypeTok{y =} \KeywordTok{log1p}\NormalTok{(mutations))) +}\StringTok{ }
\StringTok{  }\KeywordTok{geom_jitter}\NormalTok{() +}
\StringTok{  }\KeywordTok{facet_wrap}\NormalTok{(~tissue_type) +}
\StringTok{  }\KeywordTok{geom_errorbar}\NormalTok{(}\KeywordTok{aes}\NormalTok{(}\DataTypeTok{x =} \NormalTok{specific_treatment, }\DataTypeTok{ymin =} \StringTok{`}\DataTypeTok{5%}\StringTok{`}\NormalTok{, }\DataTypeTok{ymax =} \StringTok{`}\DataTypeTok{95%}\StringTok{`}\NormalTok{, }\DataTypeTok{colour =} \StringTok{'model: all samples'}\NormalTok{),}
                \DataTypeTok{data =} \NormalTok{md1 %>%}\StringTok{ }\NormalTok{dplyr::}\KeywordTok{distinct}\NormalTok{(specific_treatment, tissue_type, }\DataTypeTok{.keep_all=}\NormalTok{T),}
                \DataTypeTok{alpha =} \FloatTok{0.5}\NormalTok{) +}
\StringTok{  }\KeywordTok{geom_errorbar}\NormalTok{(}\KeywordTok{aes}\NormalTok{(}\DataTypeTok{x =} \NormalTok{specific_treatment, }\DataTypeTok{ymin =} \StringTok{`}\DataTypeTok{5%}\StringTok{`}\NormalTok{, }\DataTypeTok{ymax =} \StringTok{`}\DataTypeTok{95%}\StringTok{`}\NormalTok{, }\DataTypeTok{colour =} \StringTok{'model: recurrent only'}\NormalTok{),}
                \DataTypeTok{data =} \NormalTok{md2 %>%}\StringTok{ }\NormalTok{dplyr::}\KeywordTok{distinct}\NormalTok{(specific_treatment, tissue_type, }\DataTypeTok{.keep_all=}\NormalTok{T),}
                \DataTypeTok{alpha =} \FloatTok{0.5}\NormalTok{) +}
\StringTok{  }\KeywordTok{theme_minimal}\NormalTok{()}
\end{Highlighting}
\end{Shaded}

\includegraphics{Hierarchical_model_mutations_and_peptides_files/figure-latex/allsamp-atrt2-ppred-1.pdf}

The only explanation for this must be that the scenarios with multiple
samples per donor must have exaggerated the primary/treated vs
primary/untreated effects in this model.

What is a numerical summary of this recurrence effect, in a univariate
model adjusting only for multiple samples per donor \& tissue type?

\begin{Shaded}
\begin{Highlighting}[]
\NormalTok{rstanarm::}\KeywordTok{posterior_interval}\NormalTok{(atrt2, }\DataTypeTok{prob =} \FloatTok{0.95}\NormalTok{, }\DataTypeTok{regex_pars =} \KeywordTok{c}\NormalTok{(}\StringTok{'^timepoint'}\NormalTok{))}
\end{Highlighting}
\end{Shaded}

\begin{verbatim}
##                           2.5%     97.5%
## timepointrecurrence 0.08891223 0.6694402
\end{verbatim}

How much of this probability mass is \textless{} 0?

\begin{Shaded}
\begin{Highlighting}[]
\KeywordTok{mean}\NormalTok{(}\KeywordTok{sapply}\NormalTok{(}\KeywordTok{as.array}\NormalTok{(atrt2)[,,}\StringTok{'timepointrecurrence'}\NormalTok{], }\DataTypeTok{FUN =} \NormalTok{function(x) x <=}\StringTok{ }\DecValTok{0}\NormalTok{))}
\end{Highlighting}
\end{Shaded}

\begin{verbatim}
## [1] 0.01225
\end{verbatim}

\subsection{Analyzing peptides}\label{analyzing-peptides}

\subsubsection{Among solid samples}\label{among-solid-samples}

\begin{Shaded}
\begin{Highlighting}[]
\NormalTok{f1 <-}\StringTok{ }\NormalTok{rstanarm::}\KeywordTok{stan_glmer}\NormalTok{(}\KeywordTok{log1p}\NormalTok{(peptides) ~}\StringTok{ }
\StringTok{                                }\NormalTok{timepoint +}\StringTok{ }\NormalTok{treatment +}\StringTok{ }\NormalTok{(}\DecValTok{1} \NormalTok{|}\StringTok{ }\NormalTok{donor),}
                              \DataTypeTok{data =} \NormalTok{md_solid,}
                              \DataTypeTok{seed =} \NormalTok{stan_seed,}
                              \DataTypeTok{adapt_delta =} \FloatTok{0.999}\NormalTok{,}
                              \DataTypeTok{iter =} \DecValTok{4000}
                              \NormalTok{)}
\end{Highlighting}
\end{Shaded}

\begin{verbatim}
## Warning: There were 4 chains where the estimated Bayesian Fraction of Missing Information was low. See
## http://mc-stan.org/misc/warnings.html#bfmi-low
\end{verbatim}

\begin{verbatim}
## Warning: Examine the pairs() plot to diagnose sampling problems
\end{verbatim}

\begin{Shaded}
\begin{Highlighting}[]
\NormalTok{f1}
\end{Highlighting}
\end{Shaded}

\begin{verbatim}
## stan_glmer
##  family:  gaussian [identity]
##  formula: log1p(peptides) ~ timepoint + treatment + (1 | donor)
## ------
## 
## Estimates:
##                          Median MAD_SD
## (Intercept)              4.7    0.3   
## timepointrecurrence      0.4    0.4   
## treatmenttreatment naive 0.1    0.3   
## sigma                    0.3    0.2   
## 
## Error terms:
##  Groups   Name        Std.Dev.
##  donor    (Intercept) 0.57    
##  Residual             0.36    
## Num. levels: donor 81 
## 
## Sample avg. posterior predictive 
## distribution of y (X = xbar):
##          Median MAD_SD
## mean_PPD 4.9    0.0   
## 
## ------
## For info on the priors used see help('prior_summary.stanreg').
\end{verbatim}

\begin{Shaded}
\begin{Highlighting}[]
\KeywordTok{mean}\NormalTok{(}\KeywordTok{sapply}\NormalTok{(}\KeywordTok{as.array}\NormalTok{(f1)[,,}\StringTok{'timepointrecurrence'}\NormalTok{], }\DataTypeTok{FUN =} \NormalTok{function(x) x <=}\StringTok{ }\DecValTok{0}\NormalTok{))}
\end{Highlighting}
\end{Shaded}

\begin{verbatim}
## [1] 0.131875
\end{verbatim}

\begin{Shaded}
\begin{Highlighting}[]
\NormalTok{rstanarm::}\KeywordTok{posterior_interval}\NormalTok{(f1, }\DataTypeTok{prob =} \FloatTok{0.95}\NormalTok{, }\DataTypeTok{regex_pars =} \KeywordTok{c}\NormalTok{(}\StringTok{'^timepoint'}\NormalTok{))}
\end{Highlighting}
\end{Shaded}

\begin{verbatim}
##                           2.5%   97.5%
## timepointrecurrence -0.3180513 1.37191
\end{verbatim}

\subsubsection{Among all samples}\label{among-all-samples}

\begin{Shaded}
\begin{Highlighting}[]
\NormalTok{f2 <-}\StringTok{ }\NormalTok{rstanarm::}\KeywordTok{stan_glmer}\NormalTok{(}\KeywordTok{log1p}\NormalTok{(peptides) ~}\StringTok{ }
\StringTok{                                }\NormalTok{timepoint +}\StringTok{ }\NormalTok{treatment +}\StringTok{ }\NormalTok{(}\DecValTok{1} \NormalTok{|}\StringTok{ }\NormalTok{donor) +}
\StringTok{                                }\NormalTok{(}\DecValTok{1} \NormalTok{+}\StringTok{ }\NormalTok{timepoint |}\StringTok{ }\NormalTok{tissue_type),}
                              \DataTypeTok{data =} \NormalTok{md,}
                              \DataTypeTok{seed =} \NormalTok{stan_seed,}
                              \DataTypeTok{adapt_delta =} \FloatTok{0.999}\NormalTok{,}
                              \DataTypeTok{iter =} \DecValTok{4000}
                              \NormalTok{)}
\end{Highlighting}
\end{Shaded}

\begin{verbatim}
## Warning: There were 1 divergent transitions after warmup. Increasing adapt_delta above 0.999 may help. See
## http://mc-stan.org/misc/warnings.html#divergent-transitions-after-warmup
\end{verbatim}

\begin{verbatim}
## Warning: There were 4 chains where the estimated Bayesian Fraction of Missing Information was low. See
## http://mc-stan.org/misc/warnings.html#bfmi-low
\end{verbatim}

\begin{verbatim}
## Warning: Examine the pairs() plot to diagnose sampling problems
\end{verbatim}

\begin{Shaded}
\begin{Highlighting}[]
\NormalTok{f2}
\end{Highlighting}
\end{Shaded}

\begin{verbatim}
## stan_glmer
##  family:  gaussian [identity]
##  formula: log1p(peptides) ~ timepoint + treatment + (1 | donor) + (1 + 
##     timepoint | tissue_type)
## ------
## 
## Estimates:
##                          Median MAD_SD
## (Intercept)              4.7    0.3   
## timepointrecurrence      0.5    0.3   
## treatmenttreatment naive 0.1    0.3   
## sigma                    0.2    0.0   
## 
## Error terms:
##  Groups      Name                Std.Dev. Corr 
##  donor       (Intercept)         0.62          
##  tissue_type (Intercept)         0.19          
##              timepointrecurrence 0.19     -0.04
##  Residual                        0.24          
## Num. levels: donor 92, tissue_type 2 
## 
## Sample avg. posterior predictive 
## distribution of y (X = xbar):
##          Median MAD_SD
## mean_PPD 5.0    0.0   
## 
## ------
## For info on the priors used see help('prior_summary.stanreg').
\end{verbatim}

\begin{Shaded}
\begin{Highlighting}[]
\KeywordTok{mean}\NormalTok{(}\KeywordTok{sapply}\NormalTok{(}\KeywordTok{as.array}\NormalTok{(f2)[,,}\StringTok{'timepointrecurrence'}\NormalTok{], }\DataTypeTok{FUN =} \NormalTok{function(x) x <=}\StringTok{ }\DecValTok{0}\NormalTok{))}
\end{Highlighting}
\end{Shaded}

\begin{verbatim}
## [1] 0.086125
\end{verbatim}

\begin{Shaded}
\begin{Highlighting}[]
\NormalTok{rstanarm::}\KeywordTok{posterior_interval}\NormalTok{(f2, }\DataTypeTok{prob =} \FloatTok{0.95}\NormalTok{, }\DataTypeTok{regex_pars =} \KeywordTok{c}\NormalTok{(}\StringTok{'^timepoint'}\NormalTok{))}
\end{Highlighting}
\end{Shaded}

\begin{verbatim}
##                           2.5%    97.5%
## timepointrecurrence -0.2184951 1.098158
\end{verbatim}

\subsubsection{Plot predicted values}\label{plot-predicted-values}

\begin{Shaded}
\begin{Highlighting}[]
\NormalTok{f1.ppred <-}\StringTok{ }\NormalTok{rstanarm::}\KeywordTok{predictive_interval}\NormalTok{(f1) %>%}
\StringTok{  }\KeywordTok{tbl_df}\NormalTok{(.)}
\NormalTok{f2.ppred <-}\StringTok{ }\NormalTok{rstanarm::}\KeywordTok{predictive_interval}\NormalTok{(f2) %>%}
\StringTok{  }\KeywordTok{tbl_df}\NormalTok{(.)}

\NormalTok{f1.md <-}\StringTok{ }
\StringTok{  }\NormalTok{md_solid %>%}\StringTok{ }
\StringTok{  }\NormalTok{dplyr::}\KeywordTok{bind_cols}\NormalTok{(f1.ppred) }
\NormalTok{f2.md <-}\StringTok{ }
\StringTok{  }\NormalTok{md %>%}
\StringTok{  }\NormalTok{dplyr::}\KeywordTok{bind_cols}\NormalTok{(f2.ppred)}
\end{Highlighting}
\end{Shaded}

\begin{Shaded}
\begin{Highlighting}[]
\NormalTok{## plotting posterior-predicted values, with observed datapoints}
\KeywordTok{ggplot}\NormalTok{(md, }\KeywordTok{aes}\NormalTok{(}\DataTypeTok{x =} \NormalTok{specific_treatment, }\DataTypeTok{y =} \KeywordTok{log1p}\NormalTok{(peptides))) +}\StringTok{ }
\StringTok{  }\KeywordTok{geom_jitter}\NormalTok{() +}
\StringTok{  }\KeywordTok{facet_wrap}\NormalTok{(~tissue_type) +}
\StringTok{  }\KeywordTok{geom_errorbar}\NormalTok{(}\KeywordTok{aes}\NormalTok{(}\DataTypeTok{x =} \NormalTok{specific_treatment, }\DataTypeTok{ymin =} \StringTok{`}\DataTypeTok{5%}\StringTok{`}\NormalTok{, }\DataTypeTok{ymax =} \StringTok{`}\DataTypeTok{95%}\StringTok{`}\NormalTok{, }\DataTypeTok{colour =} \StringTok{'model: solid samples'}\NormalTok{),}
                \DataTypeTok{data =} \NormalTok{f1.md %>%}\StringTok{ }\NormalTok{dplyr::}\KeywordTok{distinct}\NormalTok{(specific_treatment, tissue_type, }\DataTypeTok{.keep_all=}\NormalTok{T),}
                \DataTypeTok{alpha =} \FloatTok{0.5}\NormalTok{) +}
\StringTok{  }\KeywordTok{geom_errorbar}\NormalTok{(}\KeywordTok{aes}\NormalTok{(}\DataTypeTok{x =} \NormalTok{specific_treatment, }\DataTypeTok{ymin =} \StringTok{`}\DataTypeTok{5%}\StringTok{`}\NormalTok{, }\DataTypeTok{ymax =} \StringTok{`}\DataTypeTok{95%}\StringTok{`}\NormalTok{, }\DataTypeTok{colour =} \StringTok{'model: all samples'}\NormalTok{),}
                \DataTypeTok{data =} \NormalTok{f2.md %>%}\StringTok{ }\NormalTok{dplyr::}\KeywordTok{distinct}\NormalTok{(specific_treatment, tissue_type, }\DataTypeTok{.keep_all=}\NormalTok{T),}
                \DataTypeTok{alpha =} \FloatTok{0.5}\NormalTok{) +}
\StringTok{  }\KeywordTok{theme_minimal}\NormalTok{()}
\end{Highlighting}
\end{Shaded}

\includegraphics{Hierarchical_model_mutations_and_peptides_files/figure-latex/allsamp-peptides-ppred-1.pdf}

\subsection{Analysis of expressed
peptides}\label{analysis-of-expressed-peptides}

\subsubsection{Among solid samples}\label{among-solid-samples-1}

\begin{Shaded}
\begin{Highlighting}[]
\NormalTok{ef1 <-}\StringTok{ }\NormalTok{rstanarm::}\KeywordTok{stan_glmer}\NormalTok{(}\KeywordTok{log1p}\NormalTok{(}\StringTok{`}\DataTypeTok{expressed peptides}\StringTok{`}\NormalTok{) ~}
\StringTok{                                }\NormalTok{timepoint +}\StringTok{ }\NormalTok{treatment +}\StringTok{ }\NormalTok{(}\DecValTok{1} \NormalTok{|}\StringTok{ }\NormalTok{donor),}
                              \DataTypeTok{data =} \NormalTok{md_solid,}
                              \DataTypeTok{seed =} \NormalTok{stan_seed,}
                              \DataTypeTok{adapt_delta =} \FloatTok{0.999}\NormalTok{,}
                              \DataTypeTok{iter =} \DecValTok{4000}
                              \NormalTok{)}
\end{Highlighting}
\end{Shaded}

\begin{verbatim}
## Warning: There were 4 chains where the estimated Bayesian Fraction of Missing Information was low. See
## http://mc-stan.org/misc/warnings.html#bfmi-low
\end{verbatim}

\begin{verbatim}
## Warning: Examine the pairs() plot to diagnose sampling problems
\end{verbatim}

\begin{Shaded}
\begin{Highlighting}[]
\NormalTok{ef1}
\end{Highlighting}
\end{Shaded}

\begin{verbatim}
## stan_glmer
##  family:  gaussian [identity]
##  formula: log1p(`expressed peptides`) ~ timepoint + treatment + (1 | donor)
## ------
## 
## Estimates:
##                          Median MAD_SD
## (Intercept)              3.2    0.3   
## timepointrecurrence      0.8    0.5   
## treatmenttreatment naive 0.6    0.3   
## sigma                    0.4    0.2   
## 
## Error terms:
##  Groups   Name        Std.Dev.
##  donor    (Intercept) 0.62    
##  Residual             0.44    
## Num. levels: donor 81 
## 
## Sample avg. posterior predictive 
## distribution of y (X = xbar):
##          Median MAD_SD
## mean_PPD 3.8    0.1   
## 
## ------
## For info on the priors used see help('prior_summary.stanreg').
\end{verbatim}

\begin{Shaded}
\begin{Highlighting}[]
\KeywordTok{mean}\NormalTok{(}\KeywordTok{sapply}\NormalTok{(}\KeywordTok{as.array}\NormalTok{(ef1)[,,}\StringTok{'timepointrecurrence'}\NormalTok{], }\DataTypeTok{FUN =} \NormalTok{function(x) x <=}\StringTok{ }\DecValTok{0}\NormalTok{))}
\end{Highlighting}
\end{Shaded}

\begin{verbatim}
## [1] 0.05175
\end{verbatim}

\begin{Shaded}
\begin{Highlighting}[]
\NormalTok{rstanarm::}\KeywordTok{posterior_interval}\NormalTok{(ef1, }\DataTypeTok{prob =} \FloatTok{0.95}\NormalTok{, }\DataTypeTok{regex_pars =} \KeywordTok{c}\NormalTok{(}\StringTok{'^timepoint'}\NormalTok{))}
\end{Highlighting}
\end{Shaded}

\begin{verbatim}
##                           2.5%    97.5%
## timepointrecurrence -0.1432569 1.859463
\end{verbatim}

\subsubsection{Among all samples}\label{among-all-samples-1}

\begin{Shaded}
\begin{Highlighting}[]
\NormalTok{ef2 <-}\StringTok{ }\NormalTok{rstanarm::}\KeywordTok{stan_glmer}\NormalTok{(}\KeywordTok{log1p}\NormalTok{(}\StringTok{`}\DataTypeTok{expressed peptides}\StringTok{`}\NormalTok{) ~}
\StringTok{                                }\NormalTok{timepoint +}\StringTok{ }\NormalTok{treatment +}\StringTok{ }\NormalTok{(}\DecValTok{1} \NormalTok{|}\StringTok{ }\NormalTok{donor) +}
\StringTok{                                }\NormalTok{(}\DecValTok{1} \NormalTok{+}\StringTok{ }\NormalTok{timepoint |}\StringTok{ }\NormalTok{tissue_type),}
                              \DataTypeTok{data =} \NormalTok{md,}
                              \DataTypeTok{seed =} \NormalTok{stan_seed,}
                              \DataTypeTok{adapt_delta =} \FloatTok{0.999}\NormalTok{,}
                              \DataTypeTok{iter =} \DecValTok{4000}
                              \NormalTok{)}
\end{Highlighting}
\end{Shaded}

\begin{verbatim}
## Warning: There were 4 chains where the estimated Bayesian Fraction of Missing Information was low. See
## http://mc-stan.org/misc/warnings.html#bfmi-low
\end{verbatim}

\begin{verbatim}
## Warning: Examine the pairs() plot to diagnose sampling problems
\end{verbatim}

\begin{Shaded}
\begin{Highlighting}[]
\NormalTok{ef2}
\end{Highlighting}
\end{Shaded}

\begin{verbatim}
## stan_glmer
##  family:  gaussian [identity]
##  formula: log1p(`expressed peptides`) ~ timepoint + treatment + (1 | donor) + 
##     (1 + timepoint | tissue_type)
## ------
## 
## Estimates:
##                          Median MAD_SD
## (Intercept)              3.3    0.4   
## timepointrecurrence      1.1    0.4   
## treatmenttreatment naive 0.6    0.3   
## sigma                    0.4    0.1   
## 
## Error terms:
##  Groups      Name                Std.Dev. Corr
##  donor       (Intercept)         0.58         
##  tissue_type (Intercept)         0.39         
##              timepointrecurrence 0.38     0.03
##  Residual                        0.45         
## Num. levels: donor 92, tissue_type 2 
## 
## Sample avg. posterior predictive 
## distribution of y (X = xbar):
##          Median MAD_SD
## mean_PPD 4.0    0.1   
## 
## ------
## For info on the priors used see help('prior_summary.stanreg').
\end{verbatim}

\begin{Shaded}
\begin{Highlighting}[]
\KeywordTok{mean}\NormalTok{(}\KeywordTok{sapply}\NormalTok{(}\KeywordTok{as.array}\NormalTok{(ef2)[,,}\StringTok{'timepointrecurrence'}\NormalTok{], }\DataTypeTok{FUN =} \NormalTok{function(x) x <=}\StringTok{ }\DecValTok{0}\NormalTok{))}
\end{Highlighting}
\end{Shaded}

\begin{verbatim}
## [1] 0.015
\end{verbatim}

\begin{Shaded}
\begin{Highlighting}[]
\NormalTok{rstanarm::}\KeywordTok{posterior_interval}\NormalTok{(ef2, }\DataTypeTok{prob =} \FloatTok{0.95}\NormalTok{, }\DataTypeTok{regex_pars =} \KeywordTok{c}\NormalTok{(}\StringTok{'^timepoint'}\NormalTok{))}
\end{Highlighting}
\end{Shaded}

\begin{verbatim}
##                          2.5%    97.5%
## timepointrecurrence 0.1573812 1.978541
\end{verbatim}

\subsubsection{Plot predicted values}\label{plot-predicted-values-1}

\begin{Shaded}
\begin{Highlighting}[]
\NormalTok{ef1.ppred <-}\StringTok{ }\NormalTok{rstanarm::}\KeywordTok{predictive_interval}\NormalTok{(ef1) %>%}
\StringTok{  }\KeywordTok{tbl_df}\NormalTok{(.)}
\NormalTok{ef2.ppred <-}\StringTok{ }\NormalTok{rstanarm::}\KeywordTok{predictive_interval}\NormalTok{(ef2) %>%}
\StringTok{  }\KeywordTok{tbl_df}\NormalTok{(.)}

\NormalTok{ef1.md <-}
\StringTok{  }\NormalTok{md_solid %>%}
\StringTok{  }\NormalTok{dplyr::}\KeywordTok{bind_cols}\NormalTok{(ef1.ppred)}
\NormalTok{ef2.md <-}
\StringTok{  }\NormalTok{md %>%}
\StringTok{  }\NormalTok{dplyr::}\KeywordTok{bind_cols}\NormalTok{(ef2.ppred)}
\end{Highlighting}
\end{Shaded}

\begin{Shaded}
\begin{Highlighting}[]
\NormalTok{## plotting posterior-predicted values, with observed datapoints}
\KeywordTok{ggplot}\NormalTok{(md, }\KeywordTok{aes}\NormalTok{(}\DataTypeTok{x =} \NormalTok{specific_treatment, }\DataTypeTok{y =} \KeywordTok{log1p}\NormalTok{(}\StringTok{`}\DataTypeTok{expressed peptides}\StringTok{`}\NormalTok{))) +}
\StringTok{  }\KeywordTok{geom_jitter}\NormalTok{() +}
\StringTok{  }\KeywordTok{facet_wrap}\NormalTok{(~tissue_type) +}
\StringTok{  }\KeywordTok{geom_errorbar}\NormalTok{(}\KeywordTok{aes}\NormalTok{(}\DataTypeTok{x =} \NormalTok{specific_treatment, }\DataTypeTok{ymin =} \StringTok{`}\DataTypeTok{5%}\StringTok{`}\NormalTok{, }\DataTypeTok{ymax =} \StringTok{`}\DataTypeTok{95%}\StringTok{`}\NormalTok{, }\DataTypeTok{colour =} \StringTok{'model: solid samples'}\NormalTok{),}
                \DataTypeTok{data =} \NormalTok{ef1.md %>%}\StringTok{ }\NormalTok{dplyr::}\KeywordTok{distinct}\NormalTok{(specific_treatment, tissue_type, }\DataTypeTok{.keep_all=}\NormalTok{T),}
                \DataTypeTok{alpha =} \FloatTok{0.5}\NormalTok{) +}
\StringTok{  }\KeywordTok{geom_errorbar}\NormalTok{(}\KeywordTok{aes}\NormalTok{(}\DataTypeTok{x =} \NormalTok{specific_treatment, }\DataTypeTok{ymin =} \StringTok{`}\DataTypeTok{5%}\StringTok{`}\NormalTok{, }\DataTypeTok{ymax =} \StringTok{`}\DataTypeTok{95%}\StringTok{`}\NormalTok{, }\DataTypeTok{colour =} \StringTok{'model: all samples'}\NormalTok{),}
                \DataTypeTok{data =} \NormalTok{ef2.md %>%}\StringTok{ }\NormalTok{dplyr::}\KeywordTok{distinct}\NormalTok{(specific_treatment, tissue_type, }\DataTypeTok{.keep_all=}\NormalTok{T),}
                \DataTypeTok{alpha =} \FloatTok{0.5}\NormalTok{) +}
\StringTok{  }\KeywordTok{theme_minimal}\NormalTok{()}
\end{Highlighting}
\end{Shaded}

\includegraphics{Hierarchical_model_mutations_and_peptides_files/figure-latex/allsamp-exppeptides-ppred-1.pdf}

\end{document}
